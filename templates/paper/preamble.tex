% =============================================================================
% preamble.tex — Shared preamble for research manuscripts
% =============================================================================
% Include via % =============================================================================
% preamble.tex — Shared preamble for research manuscripts
% =============================================================================
% Include via % =============================================================================
% preamble.tex — Shared preamble for research manuscripts
% =============================================================================
% Include via % =============================================================================
% preamble.tex — Shared preamble for research manuscripts
% =============================================================================
% Include via \input{preamble} BEFORE \begin{document}
% =============================================================================

% --- Encoding and fonts -----------------------------------------------------
\usepackage[utf8]{inputenc}
\usepackage[T1]{fontenc}
\usepackage{lmodern}                    % Latin Modern (scalable CM replacement)
\usepackage{microtype}                  % Subliminal refinements to typesetting

% --- Mathematics ------------------------------------------------------------
\usepackage{amsmath,amssymb,amsthm}
\usepackage{mathtools}                  % Extends amsmath (dcases, coloneqq, ...)
\usepackage{bm}                         % Bold math symbols via \bm{}

% --- Tables -----------------------------------------------------------------
\usepackage{booktabs}                   % \toprule, \midrule, \bottomrule
\usepackage{multirow}                   % Multi-row cells
\usepackage{array}                      % Extended column definitions

% --- Figures and graphics ---------------------------------------------------
\usepackage{graphicx}
\usepackage{xcolor}                     % Colour support (loaded before tikz)
\usepackage{subcaption}                 % Sub-figures with individual captions

% --- Units and numbers ------------------------------------------------------
\usepackage{siunitx}                    % Consistent number / unit formatting
\sisetup{
  detect-all,
  separate-uncertainty = true,
  multi-part-units     = single,
}

% --- Algorithms -------------------------------------------------------------
\usepackage[ruled,vlined,linesnumbered]{algorithm2e}

% --- Code listings (optional) -----------------------------------------------
\usepackage{listings}
\lstset{
  basicstyle   = \ttfamily\small,
  keywordstyle = \bfseries\color{blue!70!black},
  commentstyle = \itshape\color{green!50!black},
  breaklines   = true,
  frame        = single,
  numbers      = left,
  numberstyle  = \tiny\color{gray},
}

% --- Bibliography -----------------------------------------------------------
\usepackage{natbib}
\bibliographystyle{plainnat}

% --- Cross-referencing (load after most other packages) ---------------------
\usepackage[
  colorlinks  = true,
  linkcolor   = blue!60!black,
  citecolor   = green!40!black,
  urlcolor    = magenta!70!black,
]{hyperref}
\usepackage[nameinlink,capitalise,noabbrev]{cleveref}

% --- Utilities --------------------------------------------------------------
\usepackage{enumitem}                   % Customisable lists
\usepackage{xspace}                     % Smart spacing after macros
\usepackage{lipsum}                     % Dummy text (remove in final version)

% --- Custom commands --------------------------------------------------------
% Shorthand for common math operators
\DeclareMathOperator*{\argmin}{arg\,min}
\DeclareMathOperator*{\argmax}{arg\,max}
\DeclareMathOperator{\diag}{diag}
\DeclareMathOperator{\tr}{tr}
\DeclareMathOperator{\rank}{rank}
\newcommand{\given}{\,\vert\,}          % Conditional bar: P(A \given B)
\newcommand{\R}{\mathbb{R}}
\newcommand{\N}{\mathbb{N}}
\newcommand{\E}{\mathbb{E}}
\newcommand{\Var}{\operatorname{Var}}
\newcommand{\Cov}{\operatorname{Cov}}

% Convenience
\newcommand{\eg}{e.g.\@\xspace}
\newcommand{\ie}{i.e.\@\xspace}
\newcommand{\cf}{cf.\@\xspace}
\newcommand{\etal}{et~al.\@\xspace}

% TODO / note macros (disable for camera-ready by redefining to {})
\newcommand{\todo}[1]{\textcolor{red}{\textbf{[TODO:} #1\textbf{]}}}
\newcommand{\note}[1]{\textcolor{blue}{\textbf{[NOTE:} #1\textbf{]}}}

% Theorem-like environments
\theoremstyle{plain}
\newtheorem{theorem}{Theorem}[section]
\newtheorem{lemma}[theorem]{Lemma}
\newtheorem{proposition}[theorem]{Proposition}
\newtheorem{corollary}[theorem]{Corollary}
\theoremstyle{definition}
\newtheorem{definition}[theorem]{Definition}
\theoremstyle{remark}
\newtheorem{remark}[theorem]{Remark}

% =============================================================================
% End of preamble
% =============================================================================
 BEFORE \begin{document}
% =============================================================================

% --- Encoding and fonts -----------------------------------------------------
\usepackage[utf8]{inputenc}
\usepackage[T1]{fontenc}
\usepackage{lmodern}                    % Latin Modern (scalable CM replacement)
\usepackage{microtype}                  % Subliminal refinements to typesetting

% --- Mathematics ------------------------------------------------------------
\usepackage{amsmath,amssymb,amsthm}
\usepackage{mathtools}                  % Extends amsmath (dcases, coloneqq, ...)
\usepackage{bm}                         % Bold math symbols via \bm{}

% --- Tables -----------------------------------------------------------------
\usepackage{booktabs}                   % \toprule, \midrule, \bottomrule
\usepackage{multirow}                   % Multi-row cells
\usepackage{array}                      % Extended column definitions

% --- Figures and graphics ---------------------------------------------------
\usepackage{graphicx}
\usepackage{xcolor}                     % Colour support (loaded before tikz)
\usepackage{subcaption}                 % Sub-figures with individual captions

% --- Units and numbers ------------------------------------------------------
\usepackage{siunitx}                    % Consistent number / unit formatting
\sisetup{
  detect-all,
  separate-uncertainty = true,
  multi-part-units     = single,
}

% --- Algorithms -------------------------------------------------------------
\usepackage[ruled,vlined,linesnumbered]{algorithm2e}

% --- Code listings (optional) -----------------------------------------------
\usepackage{listings}
\lstset{
  basicstyle   = \ttfamily\small,
  keywordstyle = \bfseries\color{blue!70!black},
  commentstyle = \itshape\color{green!50!black},
  breaklines   = true,
  frame        = single,
  numbers      = left,
  numberstyle  = \tiny\color{gray},
}

% --- Bibliography -----------------------------------------------------------
\usepackage{natbib}
\bibliographystyle{plainnat}

% --- Cross-referencing (load after most other packages) ---------------------
\usepackage[
  colorlinks  = true,
  linkcolor   = blue!60!black,
  citecolor   = green!40!black,
  urlcolor    = magenta!70!black,
]{hyperref}
\usepackage[nameinlink,capitalise,noabbrev]{cleveref}

% --- Utilities --------------------------------------------------------------
\usepackage{enumitem}                   % Customisable lists
\usepackage{xspace}                     % Smart spacing after macros
\usepackage{lipsum}                     % Dummy text (remove in final version)

% --- Custom commands --------------------------------------------------------
% Shorthand for common math operators
\DeclareMathOperator*{\argmin}{arg\,min}
\DeclareMathOperator*{\argmax}{arg\,max}
\DeclareMathOperator{\diag}{diag}
\DeclareMathOperator{\tr}{tr}
\DeclareMathOperator{\rank}{rank}
\newcommand{\given}{\,\vert\,}          % Conditional bar: P(A \given B)
\newcommand{\R}{\mathbb{R}}
\newcommand{\N}{\mathbb{N}}
\newcommand{\E}{\mathbb{E}}
\newcommand{\Var}{\operatorname{Var}}
\newcommand{\Cov}{\operatorname{Cov}}

% Convenience
\newcommand{\eg}{e.g.\@\xspace}
\newcommand{\ie}{i.e.\@\xspace}
\newcommand{\cf}{cf.\@\xspace}
\newcommand{\etal}{et~al.\@\xspace}

% TODO / note macros (disable for camera-ready by redefining to {})
\newcommand{\todo}[1]{\textcolor{red}{\textbf{[TODO:} #1\textbf{]}}}
\newcommand{\note}[1]{\textcolor{blue}{\textbf{[NOTE:} #1\textbf{]}}}

% Theorem-like environments
\theoremstyle{plain}
\newtheorem{theorem}{Theorem}[section]
\newtheorem{lemma}[theorem]{Lemma}
\newtheorem{proposition}[theorem]{Proposition}
\newtheorem{corollary}[theorem]{Corollary}
\theoremstyle{definition}
\newtheorem{definition}[theorem]{Definition}
\theoremstyle{remark}
\newtheorem{remark}[theorem]{Remark}

% =============================================================================
% End of preamble
% =============================================================================
 BEFORE \begin{document}
% =============================================================================

% --- Encoding and fonts -----------------------------------------------------
\usepackage[utf8]{inputenc}
\usepackage[T1]{fontenc}
\usepackage{lmodern}                    % Latin Modern (scalable CM replacement)
\usepackage{microtype}                  % Subliminal refinements to typesetting

% --- Mathematics ------------------------------------------------------------
\usepackage{amsmath,amssymb,amsthm}
\usepackage{mathtools}                  % Extends amsmath (dcases, coloneqq, ...)
\usepackage{bm}                         % Bold math symbols via \bm{}

% --- Tables -----------------------------------------------------------------
\usepackage{booktabs}                   % \toprule, \midrule, \bottomrule
\usepackage{multirow}                   % Multi-row cells
\usepackage{array}                      % Extended column definitions

% --- Figures and graphics ---------------------------------------------------
\usepackage{graphicx}
\usepackage{xcolor}                     % Colour support (loaded before tikz)
\usepackage{subcaption}                 % Sub-figures with individual captions

% --- Units and numbers ------------------------------------------------------
\usepackage{siunitx}                    % Consistent number / unit formatting
\sisetup{
  detect-all,
  separate-uncertainty = true,
  multi-part-units     = single,
}

% --- Algorithms -------------------------------------------------------------
\usepackage[ruled,vlined,linesnumbered]{algorithm2e}

% --- Code listings (optional) -----------------------------------------------
\usepackage{listings}
\lstset{
  basicstyle   = \ttfamily\small,
  keywordstyle = \bfseries\color{blue!70!black},
  commentstyle = \itshape\color{green!50!black},
  breaklines   = true,
  frame        = single,
  numbers      = left,
  numberstyle  = \tiny\color{gray},
}

% --- Bibliography -----------------------------------------------------------
\usepackage{natbib}
\bibliographystyle{plainnat}

% --- Cross-referencing (load after most other packages) ---------------------
\usepackage[
  colorlinks  = true,
  linkcolor   = blue!60!black,
  citecolor   = green!40!black,
  urlcolor    = magenta!70!black,
]{hyperref}
\usepackage[nameinlink,capitalise,noabbrev]{cleveref}

% --- Utilities --------------------------------------------------------------
\usepackage{enumitem}                   % Customisable lists
\usepackage{xspace}                     % Smart spacing after macros
\usepackage{lipsum}                     % Dummy text (remove in final version)

% --- Custom commands --------------------------------------------------------
% Shorthand for common math operators
\DeclareMathOperator*{\argmin}{arg\,min}
\DeclareMathOperator*{\argmax}{arg\,max}
\DeclareMathOperator{\diag}{diag}
\DeclareMathOperator{\tr}{tr}
\DeclareMathOperator{\rank}{rank}
\newcommand{\given}{\,\vert\,}          % Conditional bar: P(A \given B)
\newcommand{\R}{\mathbb{R}}
\newcommand{\N}{\mathbb{N}}
\newcommand{\E}{\mathbb{E}}
\newcommand{\Var}{\operatorname{Var}}
\newcommand{\Cov}{\operatorname{Cov}}

% Convenience
\newcommand{\eg}{e.g.\@\xspace}
\newcommand{\ie}{i.e.\@\xspace}
\newcommand{\cf}{cf.\@\xspace}
\newcommand{\etal}{et~al.\@\xspace}

% TODO / note macros (disable for camera-ready by redefining to {})
\newcommand{\todo}[1]{\textcolor{red}{\textbf{[TODO:} #1\textbf{]}}}
\newcommand{\note}[1]{\textcolor{blue}{\textbf{[NOTE:} #1\textbf{]}}}

% Theorem-like environments
\theoremstyle{plain}
\newtheorem{theorem}{Theorem}[section]
\newtheorem{lemma}[theorem]{Lemma}
\newtheorem{proposition}[theorem]{Proposition}
\newtheorem{corollary}[theorem]{Corollary}
\theoremstyle{definition}
\newtheorem{definition}[theorem]{Definition}
\theoremstyle{remark}
\newtheorem{remark}[theorem]{Remark}

% =============================================================================
% End of preamble
% =============================================================================
 BEFORE \begin{document}
% =============================================================================

% --- Encoding and fonts -----------------------------------------------------
\usepackage[utf8]{inputenc}
\usepackage[T1]{fontenc}
\usepackage{lmodern}                    % Latin Modern (scalable CM replacement)
\usepackage{microtype}                  % Subliminal refinements to typesetting

% --- Mathematics ------------------------------------------------------------
\usepackage{amsmath,amssymb,amsthm}
\usepackage{mathtools}                  % Extends amsmath (dcases, coloneqq, ...)
\usepackage{bm}                         % Bold math symbols via \bm{}

% --- Tables -----------------------------------------------------------------
\usepackage{booktabs}                   % \toprule, \midrule, \bottomrule
\usepackage{multirow}                   % Multi-row cells
\usepackage{array}                      % Extended column definitions

% --- Figures and graphics ---------------------------------------------------
\usepackage{graphicx}
\usepackage{xcolor}                     % Colour support (loaded before tikz)
\usepackage{subcaption}                 % Sub-figures with individual captions

% --- Units and numbers ------------------------------------------------------
\usepackage{siunitx}                    % Consistent number / unit formatting
\sisetup{
  detect-all,
  separate-uncertainty = true,
  multi-part-units     = single,
}

% --- Algorithms -------------------------------------------------------------
\usepackage[ruled,vlined,linesnumbered]{algorithm2e}

% --- Code listings (optional) -----------------------------------------------
\usepackage{listings}
\lstset{
  basicstyle   = \ttfamily\small,
  keywordstyle = \bfseries\color{blue!70!black},
  commentstyle = \itshape\color{green!50!black},
  breaklines   = true,
  frame        = single,
  numbers      = left,
  numberstyle  = \tiny\color{gray},
}

% --- Bibliography -----------------------------------------------------------
\usepackage{natbib}
\bibliographystyle{plainnat}

% --- Cross-referencing (load after most other packages) ---------------------
\usepackage[
  colorlinks  = true,
  linkcolor   = blue!60!black,
  citecolor   = green!40!black,
  urlcolor    = magenta!70!black,
]{hyperref}
\usepackage[nameinlink,capitalise,noabbrev]{cleveref}

% --- Utilities --------------------------------------------------------------
\usepackage{enumitem}                   % Customisable lists
\usepackage{xspace}                     % Smart spacing after macros
\usepackage{lipsum}                     % Dummy text (remove in final version)

% --- Custom commands --------------------------------------------------------
% Shorthand for common math operators
\DeclareMathOperator*{\argmin}{arg\,min}
\DeclareMathOperator*{\argmax}{arg\,max}
\DeclareMathOperator{\diag}{diag}
\DeclareMathOperator{\tr}{tr}
\DeclareMathOperator{\rank}{rank}
\newcommand{\given}{\,\vert\,}          % Conditional bar: P(A \given B)
\newcommand{\R}{\mathbb{R}}
\newcommand{\N}{\mathbb{N}}
\newcommand{\E}{\mathbb{E}}
\newcommand{\Var}{\operatorname{Var}}
\newcommand{\Cov}{\operatorname{Cov}}

% Convenience
\newcommand{\eg}{e.g.\@\xspace}
\newcommand{\ie}{i.e.\@\xspace}
\newcommand{\cf}{cf.\@\xspace}
\newcommand{\etal}{et~al.\@\xspace}

% TODO / note macros (disable for camera-ready by redefining to {})
\newcommand{\todo}[1]{\textcolor{red}{\textbf{[TODO:} #1\textbf{]}}}
\newcommand{\note}[1]{\textcolor{blue}{\textbf{[NOTE:} #1\textbf{]}}}

% Theorem-like environments
\theoremstyle{plain}
\newtheorem{theorem}{Theorem}[section]
\newtheorem{lemma}[theorem]{Lemma}
\newtheorem{proposition}[theorem]{Proposition}
\newtheorem{corollary}[theorem]{Corollary}
\theoremstyle{definition}
\newtheorem{definition}[theorem]{Definition}
\theoremstyle{remark}
\newtheorem{remark}[theorem]{Remark}

% =============================================================================
% End of preamble
% =============================================================================
