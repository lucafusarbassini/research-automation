% =============================================================================
%  main.tex — Research manuscript template
% =============================================================================
%
%  Build:   make all        (or: latexmk -pdf main.tex)
%  Clean:   make clean
%  Watch:   make watch
%
% =============================================================================
\documentclass[11pt,a4paper,onecolumn]{article}

% --- Shared preamble --------------------------------------------------------
% =============================================================================
% preamble.tex — Shared preamble for research manuscripts
% =============================================================================
% Include via % =============================================================================
% preamble.tex — Shared preamble for research manuscripts
% =============================================================================
% Include via % =============================================================================
% preamble.tex — Shared preamble for research manuscripts
% =============================================================================
% Include via \input{preamble} BEFORE \begin{document}
% =============================================================================

% --- Encoding and fonts -----------------------------------------------------
\usepackage[utf8]{inputenc}
\usepackage[T1]{fontenc}
\usepackage{lmodern}                    % Latin Modern (scalable CM replacement)
\usepackage{microtype}                  % Subliminal refinements to typesetting

% --- Mathematics ------------------------------------------------------------
\usepackage{amsmath,amssymb,amsthm}
\usepackage{mathtools}                  % Extends amsmath (dcases, coloneqq, ...)
\usepackage{bm}                         % Bold math symbols via \bm{}

% --- Tables -----------------------------------------------------------------
\usepackage{booktabs}                   % \toprule, \midrule, \bottomrule
\usepackage{multirow}                   % Multi-row cells
\usepackage{array}                      % Extended column definitions

% --- Figures and graphics ---------------------------------------------------
\usepackage{graphicx}
\usepackage{xcolor}                     % Colour support (loaded before tikz)
\usepackage{subcaption}                 % Sub-figures with individual captions

% --- Units and numbers ------------------------------------------------------
\usepackage{siunitx}                    % Consistent number / unit formatting
\sisetup{
  detect-all,
  separate-uncertainty = true,
  multi-part-units     = single,
}

% --- Algorithms -------------------------------------------------------------
\usepackage[ruled,vlined,linesnumbered]{algorithm2e}

% --- Code listings (optional) -----------------------------------------------
\usepackage{listings}
\lstset{
  basicstyle   = \ttfamily\small,
  keywordstyle = \bfseries\color{blue!70!black},
  commentstyle = \itshape\color{green!50!black},
  breaklines   = true,
  frame        = single,
  numbers      = left,
  numberstyle  = \tiny\color{gray},
}

% --- Bibliography -----------------------------------------------------------
\usepackage{natbib}
\bibliographystyle{plainnat}

% --- Cross-referencing (load after most other packages) ---------------------
\usepackage[
  colorlinks  = true,
  linkcolor   = blue!60!black,
  citecolor   = green!40!black,
  urlcolor    = magenta!70!black,
]{hyperref}
\usepackage[nameinlink,capitalise,noabbrev]{cleveref}

% --- Utilities --------------------------------------------------------------
\usepackage{enumitem}                   % Customisable lists
\usepackage{xspace}                     % Smart spacing after macros
\usepackage{lipsum}                     % Dummy text (remove in final version)

% --- Custom commands --------------------------------------------------------
% Shorthand for common math operators
\DeclareMathOperator*{\argmin}{arg\,min}
\DeclareMathOperator*{\argmax}{arg\,max}
\DeclareMathOperator{\diag}{diag}
\DeclareMathOperator{\tr}{tr}
\DeclareMathOperator{\rank}{rank}
\newcommand{\given}{\,\vert\,}          % Conditional bar: P(A \given B)
\newcommand{\R}{\mathbb{R}}
\newcommand{\N}{\mathbb{N}}
\newcommand{\E}{\mathbb{E}}
\newcommand{\Var}{\operatorname{Var}}
\newcommand{\Cov}{\operatorname{Cov}}

% Convenience
\newcommand{\eg}{e.g.\@\xspace}
\newcommand{\ie}{i.e.\@\xspace}
\newcommand{\cf}{cf.\@\xspace}
\newcommand{\etal}{et~al.\@\xspace}

% TODO / note macros (disable for camera-ready by redefining to {})
\newcommand{\todo}[1]{\textcolor{red}{\textbf{[TODO:} #1\textbf{]}}}
\newcommand{\note}[1]{\textcolor{blue}{\textbf{[NOTE:} #1\textbf{]}}}

% Theorem-like environments
\theoremstyle{plain}
\newtheorem{theorem}{Theorem}[section]
\newtheorem{lemma}[theorem]{Lemma}
\newtheorem{proposition}[theorem]{Proposition}
\newtheorem{corollary}[theorem]{Corollary}
\theoremstyle{definition}
\newtheorem{definition}[theorem]{Definition}
\theoremstyle{remark}
\newtheorem{remark}[theorem]{Remark}

% =============================================================================
% End of preamble
% =============================================================================
 BEFORE \begin{document}
% =============================================================================

% --- Encoding and fonts -----------------------------------------------------
\usepackage[utf8]{inputenc}
\usepackage[T1]{fontenc}
\usepackage{lmodern}                    % Latin Modern (scalable CM replacement)
\usepackage{microtype}                  % Subliminal refinements to typesetting

% --- Mathematics ------------------------------------------------------------
\usepackage{amsmath,amssymb,amsthm}
\usepackage{mathtools}                  % Extends amsmath (dcases, coloneqq, ...)
\usepackage{bm}                         % Bold math symbols via \bm{}

% --- Tables -----------------------------------------------------------------
\usepackage{booktabs}                   % \toprule, \midrule, \bottomrule
\usepackage{multirow}                   % Multi-row cells
\usepackage{array}                      % Extended column definitions

% --- Figures and graphics ---------------------------------------------------
\usepackage{graphicx}
\usepackage{xcolor}                     % Colour support (loaded before tikz)
\usepackage{subcaption}                 % Sub-figures with individual captions

% --- Units and numbers ------------------------------------------------------
\usepackage{siunitx}                    % Consistent number / unit formatting
\sisetup{
  detect-all,
  separate-uncertainty = true,
  multi-part-units     = single,
}

% --- Algorithms -------------------------------------------------------------
\usepackage[ruled,vlined,linesnumbered]{algorithm2e}

% --- Code listings (optional) -----------------------------------------------
\usepackage{listings}
\lstset{
  basicstyle   = \ttfamily\small,
  keywordstyle = \bfseries\color{blue!70!black},
  commentstyle = \itshape\color{green!50!black},
  breaklines   = true,
  frame        = single,
  numbers      = left,
  numberstyle  = \tiny\color{gray},
}

% --- Bibliography -----------------------------------------------------------
\usepackage{natbib}
\bibliographystyle{plainnat}

% --- Cross-referencing (load after most other packages) ---------------------
\usepackage[
  colorlinks  = true,
  linkcolor   = blue!60!black,
  citecolor   = green!40!black,
  urlcolor    = magenta!70!black,
]{hyperref}
\usepackage[nameinlink,capitalise,noabbrev]{cleveref}

% --- Utilities --------------------------------------------------------------
\usepackage{enumitem}                   % Customisable lists
\usepackage{xspace}                     % Smart spacing after macros
\usepackage{lipsum}                     % Dummy text (remove in final version)

% --- Custom commands --------------------------------------------------------
% Shorthand for common math operators
\DeclareMathOperator*{\argmin}{arg\,min}
\DeclareMathOperator*{\argmax}{arg\,max}
\DeclareMathOperator{\diag}{diag}
\DeclareMathOperator{\tr}{tr}
\DeclareMathOperator{\rank}{rank}
\newcommand{\given}{\,\vert\,}          % Conditional bar: P(A \given B)
\newcommand{\R}{\mathbb{R}}
\newcommand{\N}{\mathbb{N}}
\newcommand{\E}{\mathbb{E}}
\newcommand{\Var}{\operatorname{Var}}
\newcommand{\Cov}{\operatorname{Cov}}

% Convenience
\newcommand{\eg}{e.g.\@\xspace}
\newcommand{\ie}{i.e.\@\xspace}
\newcommand{\cf}{cf.\@\xspace}
\newcommand{\etal}{et~al.\@\xspace}

% TODO / note macros (disable for camera-ready by redefining to {})
\newcommand{\todo}[1]{\textcolor{red}{\textbf{[TODO:} #1\textbf{]}}}
\newcommand{\note}[1]{\textcolor{blue}{\textbf{[NOTE:} #1\textbf{]}}}

% Theorem-like environments
\theoremstyle{plain}
\newtheorem{theorem}{Theorem}[section]
\newtheorem{lemma}[theorem]{Lemma}
\newtheorem{proposition}[theorem]{Proposition}
\newtheorem{corollary}[theorem]{Corollary}
\theoremstyle{definition}
\newtheorem{definition}[theorem]{Definition}
\theoremstyle{remark}
\newtheorem{remark}[theorem]{Remark}

% =============================================================================
% End of preamble
% =============================================================================
 BEFORE \begin{document}
% =============================================================================

% --- Encoding and fonts -----------------------------------------------------
\usepackage[utf8]{inputenc}
\usepackage[T1]{fontenc}
\usepackage{lmodern}                    % Latin Modern (scalable CM replacement)
\usepackage{microtype}                  % Subliminal refinements to typesetting

% --- Mathematics ------------------------------------------------------------
\usepackage{amsmath,amssymb,amsthm}
\usepackage{mathtools}                  % Extends amsmath (dcases, coloneqq, ...)
\usepackage{bm}                         % Bold math symbols via \bm{}

% --- Tables -----------------------------------------------------------------
\usepackage{booktabs}                   % \toprule, \midrule, \bottomrule
\usepackage{multirow}                   % Multi-row cells
\usepackage{array}                      % Extended column definitions

% --- Figures and graphics ---------------------------------------------------
\usepackage{graphicx}
\usepackage{xcolor}                     % Colour support (loaded before tikz)
\usepackage{subcaption}                 % Sub-figures with individual captions

% --- Units and numbers ------------------------------------------------------
\usepackage{siunitx}                    % Consistent number / unit formatting
\sisetup{
  detect-all,
  separate-uncertainty = true,
  multi-part-units     = single,
}

% --- Algorithms -------------------------------------------------------------
\usepackage[ruled,vlined,linesnumbered]{algorithm2e}

% --- Code listings (optional) -----------------------------------------------
\usepackage{listings}
\lstset{
  basicstyle   = \ttfamily\small,
  keywordstyle = \bfseries\color{blue!70!black},
  commentstyle = \itshape\color{green!50!black},
  breaklines   = true,
  frame        = single,
  numbers      = left,
  numberstyle  = \tiny\color{gray},
}

% --- Bibliography -----------------------------------------------------------
\usepackage{natbib}
\bibliographystyle{plainnat}

% --- Cross-referencing (load after most other packages) ---------------------
\usepackage[
  colorlinks  = true,
  linkcolor   = blue!60!black,
  citecolor   = green!40!black,
  urlcolor    = magenta!70!black,
]{hyperref}
\usepackage[nameinlink,capitalise,noabbrev]{cleveref}

% --- Utilities --------------------------------------------------------------
\usepackage{enumitem}                   % Customisable lists
\usepackage{xspace}                     % Smart spacing after macros
\usepackage{lipsum}                     % Dummy text (remove in final version)

% --- Custom commands --------------------------------------------------------
% Shorthand for common math operators
\DeclareMathOperator*{\argmin}{arg\,min}
\DeclareMathOperator*{\argmax}{arg\,max}
\DeclareMathOperator{\diag}{diag}
\DeclareMathOperator{\tr}{tr}
\DeclareMathOperator{\rank}{rank}
\newcommand{\given}{\,\vert\,}          % Conditional bar: P(A \given B)
\newcommand{\R}{\mathbb{R}}
\newcommand{\N}{\mathbb{N}}
\newcommand{\E}{\mathbb{E}}
\newcommand{\Var}{\operatorname{Var}}
\newcommand{\Cov}{\operatorname{Cov}}

% Convenience
\newcommand{\eg}{e.g.\@\xspace}
\newcommand{\ie}{i.e.\@\xspace}
\newcommand{\cf}{cf.\@\xspace}
\newcommand{\etal}{et~al.\@\xspace}

% TODO / note macros (disable for camera-ready by redefining to {})
\newcommand{\todo}[1]{\textcolor{red}{\textbf{[TODO:} #1\textbf{]}}}
\newcommand{\note}[1]{\textcolor{blue}{\textbf{[NOTE:} #1\textbf{]}}}

% Theorem-like environments
\theoremstyle{plain}
\newtheorem{theorem}{Theorem}[section]
\newtheorem{lemma}[theorem]{Lemma}
\newtheorem{proposition}[theorem]{Proposition}
\newtheorem{corollary}[theorem]{Corollary}
\theoremstyle{definition}
\newtheorem{definition}[theorem]{Definition}
\theoremstyle{remark}
\newtheorem{remark}[theorem]{Remark}

% =============================================================================
% End of preamble
% =============================================================================


% --- Page geometry ----------------------------------------------------------
\usepackage[
  top    = 2.5cm,
  bottom = 2.5cm,
  left   = 2.5cm,
  right  = 2.5cm,
]{geometry}

% --- Line spacing (adjust for submission) -----------------------------------
\usepackage{setspace}
\onehalfspacing           % Switch to \doublespacing for review copies

% --- Line numbers (uncomment for review) ------------------------------------
% \usepackage{lineno}
% \linenumbers

% =============================================================================
%  Metadata
% =============================================================================
\title{%
  \textbf{A Descriptive Title That Clearly Communicates\\
  the Main Finding of This Study}%
}

\author{%
  First~Author\textsuperscript{1,*},\quad
  Second~Author\textsuperscript{2},\quad
  Third~Author\textsuperscript{1,2}\\[6pt]
  \small\textsuperscript{1}Department of Example, University of Somewhere,
    City, Country\\
  \small\textsuperscript{2}Institute of Research, Organisation, City, Country\\[4pt]
  \small\textsuperscript{*}Corresponding author:
    \href{mailto:first.author@example.com}{first.author@example.com}
}

\date{}  % Suppress date; journals set their own

% =============================================================================
\begin{document}
% =============================================================================

\maketitle
\thispagestyle{empty}  % No page number on title page

% --- Abstract ---------------------------------------------------------------
\begin{abstract}
\noindent
\textbf{Background.}\quad
Provide context and motivation for the study in one to two sentences.
%
\textbf{Methods.}\quad
Briefly describe the experimental or computational approach.
%
\textbf{Results.}\quad
State the key findings with quantitative detail where possible.
%
\textbf{Conclusions.}\quad
Summarise the implications and significance of the work.

\medskip
\noindent
\textbf{Keywords:}\quad
keyword one, keyword two, keyword three, keyword four, keyword five
\end{abstract}

\clearpage

% =============================================================================
\section{Introduction}
\label{sec:introduction}
% =============================================================================

% Opening paragraph: broad context and significance of the problem.

% Second paragraph: what is currently known, with key citations
% \citep{AuthorYear}.

% Third paragraph: the gap in knowledge or the unresolved question.

% Final paragraph: state the aim and outline the approach taken.
% ``In this work, we \ldots''

\todo{Write the introduction.}

% =============================================================================
\section{Methods}
\label{sec:methods}
% =============================================================================

\subsection{Data Collection}
\label{sec:methods:data}

% Describe datasets, cohorts, or experimental materials.

\subsection{Computational Pipeline}
\label{sec:methods:pipeline}

% Describe the algorithms, models, or statistical methods.

\subsection{Statistical Analysis}
\label{sec:methods:statistics}

% Describe significance tests, multiple-testing correction, etc.

\todo{Write the methods.}

% =============================================================================
\section{Results}
\label{sec:results}
% =============================================================================

\subsection{Descriptive Overview}
\label{sec:results:overview}

% Present the first set of findings with references to figures and tables.
% Use \cref{fig:example} and \cref{tab:example} for cross-references.

\subsection{Key Finding}
\label{sec:results:key}

% Present the central result with quantitative evidence.

\todo{Write the results.}

% --- Example figure ---------------------------------------------------------
% \begin{figure}[htbp]
%   \centering
%   \includegraphics[width=0.8\textwidth]{figures/placeholder.pdf}
%   \caption{%
%     \textbf{Short descriptive title.}
%     Detailed explanation of what the figure shows, including axis labels
%     and any statistical annotations.
%   }
%   \label{fig:example}
% \end{figure}

% --- Example table ----------------------------------------------------------
% \begin{table}[htbp]
%   \centering
%   \caption{%
%     \textbf{Summary statistics.}
%     Description of the table contents and any relevant notes.
%   }
%   \label{tab:example}
%   \begin{tabular}{@{} l S[table-format=4.1] S[table-format=1.3] @{}}
%     \toprule
%     {Method} & {Score} & {$p$-value} \\
%     \midrule
%     Baseline   & 72.3  & 0.041 \\
%     Our method & 89.7  & 0.001 \\
%     \bottomrule
%   \end{tabular}
% \end{table}

% =============================================================================
\section{Discussion}
\label{sec:discussion}
% =============================================================================

% Paragraph 1: Restate the main finding and its significance.

% Paragraph 2: How does this compare with prior work?

% Paragraph 3: Limitations and potential confounders.

% Paragraph 4: Future directions.

\todo{Write the discussion.}

% =============================================================================
\section{Conclusion}
\label{sec:conclusion}
% =============================================================================

% A concise summary (one paragraph) of the contribution and outlook.

\todo{Write the conclusion.}

% =============================================================================
%  Back matter
% =============================================================================

\section*{Data Availability}

% State where data and code can be accessed (e.g., Zenodo, GitHub).

\section*{Author Contributions}

% CRediT taxonomy: Conceptualization, Methodology, Software, Validation,
% Formal analysis, Investigation, Resources, Data curation, Writing --
% original draft, Writing -- review \& editing, Visualisation, Supervision,
% Project administration, Funding acquisition.

\section*{Acknowledgements}

% Funding agencies, compute resources, helpful discussions.

\section*{Conflicts of Interest}

The authors declare no competing interests.

% --- Bibliography -----------------------------------------------------------
\bibliography{references}

% --- Supplementary (separate file) ------------------------------------------
% \clearpage
% % =============================================================================
%  supplementary.tex — Supplementary Materials
% =============================================================================
%
%  Usage: % =============================================================================
%  supplementary.tex — Supplementary Materials
% =============================================================================
%
%  Usage: % =============================================================================
%  supplementary.tex — Supplementary Materials
% =============================================================================
%
%  Usage: \input{supplementary} at the end of main.tex,
%         or compile standalone (uncomment the preamble block below).
%
% =============================================================================

% --- Uncomment the following block to compile as a standalone document ------
% \documentclass[11pt,a4paper]{article}
% \input{preamble}
% \usepackage[top=2.5cm,bottom=2.5cm,left=2.5cm,right=2.5cm]{geometry}
% \usepackage{setspace}\onehalfspacing
% \title{Supplementary Materials}
% \author{}
% \date{}
% \begin{document}
% \maketitle
% ---

% =============================================================================
%  Reset counters for supplementary numbering (S1, S2, ...)
% =============================================================================
\setcounter{section}{0}
\setcounter{figure}{0}
\setcounter{table}{0}
\setcounter{equation}{0}

\renewcommand{\thesection}{S\arabic{section}}
\renewcommand{\thefigure}{S\arabic{figure}}
\renewcommand{\thetable}{S\arabic{table}}
\renewcommand{\theequation}{S\arabic{equation}}

% Update cleveref names
\crefname{section}{Supplementary Section}{Supplementary Sections}
\crefname{figure}{Supplementary Figure}{Supplementary Figures}
\crefname{table}{Supplementary Table}{Supplementary Tables}
\crefname{equation}{Supplementary Equation}{Supplementary Equations}

\clearpage
\begin{center}
  {\LARGE\bfseries Supplementary Materials}\\[12pt]
  {\large A Descriptive Title That Clearly Communicates
   the Main Finding of This Study}
\end{center}

\bigskip
\tableofcontents

% =============================================================================
\section{Supplementary Methods}
\label{sec:supp:methods}
% =============================================================================

% Extended methodological detail that does not fit in the main text.
% Include derivations, parameter choices, preprocessing steps, etc.

\subsection{Parameter Selection}
\label{sec:supp:methods:params}

% Describe hyperparameter tuning, cross-validation, grid search, etc.

\subsection{Detailed Derivations}
\label{sec:supp:methods:derivations}

% Full mathematical derivations for key results.

% =============================================================================
\section{Supplementary Figures}
\label{sec:supp:figures}
% =============================================================================

% \begin{figure}[htbp]
%   \centering
%   \includegraphics[width=0.85\textwidth]{figures/supp_placeholder.pdf}
%   \caption{%
%     \textbf{Supplementary Figure S1: Extended analysis.}
%     Detailed caption describing what the figure shows, how it was generated,
%     and how to interpret it.
%   }
%   \label{fig:supp:example}
% \end{figure}

% =============================================================================
\section{Supplementary Tables}
\label{sec:supp:tables}
% =============================================================================

% \begin{table}[htbp]
%   \centering
%   \caption{%
%     \textbf{Supplementary Table S1: Full benchmark results.}
%     Complete results for all methods and datasets.
%   }
%   \label{tab:supp:benchmark}
%   \begin{tabular}{@{} l *{4}{S[table-format=2.1]} @{}}
%     \toprule
%     {Method} & {Dataset A} & {Dataset B} & {Dataset C} & {Dataset D} \\
%     \midrule
%     Baseline   & 72.3 & 68.1 & 75.4 & 70.9 \\
%     Variant 1  & 78.5 & 74.2 & 80.1 & 76.3 \\
%     Variant 2  & 81.2 & 77.8 & 83.6 & 79.1 \\
%     Our method & 89.7 & 85.4 & 91.2 & 87.5 \\
%     \bottomrule
%   \end{tabular}
% \end{table}

% =============================================================================
\section{Supplementary Notes}
\label{sec:supp:notes}
% =============================================================================

% Additional discussion, edge cases, or extended analysis.

% --- Uncomment if compiling standalone --------------------------------------
% \end{document}
 at the end of main.tex,
%         or compile standalone (uncomment the preamble block below).
%
% =============================================================================

% --- Uncomment the following block to compile as a standalone document ------
% \documentclass[11pt,a4paper]{article}
% % =============================================================================
% preamble.tex — Shared preamble for research manuscripts
% =============================================================================
% Include via \input{preamble} BEFORE \begin{document}
% =============================================================================

% --- Encoding and fonts -----------------------------------------------------
\usepackage[utf8]{inputenc}
\usepackage[T1]{fontenc}
\usepackage{lmodern}                    % Latin Modern (scalable CM replacement)
\usepackage{microtype}                  % Subliminal refinements to typesetting

% --- Mathematics ------------------------------------------------------------
\usepackage{amsmath,amssymb,amsthm}
\usepackage{mathtools}                  % Extends amsmath (dcases, coloneqq, ...)
\usepackage{bm}                         % Bold math symbols via \bm{}

% --- Tables -----------------------------------------------------------------
\usepackage{booktabs}                   % \toprule, \midrule, \bottomrule
\usepackage{multirow}                   % Multi-row cells
\usepackage{array}                      % Extended column definitions

% --- Figures and graphics ---------------------------------------------------
\usepackage{graphicx}
\usepackage{xcolor}                     % Colour support (loaded before tikz)
\usepackage{subcaption}                 % Sub-figures with individual captions

% --- Units and numbers ------------------------------------------------------
\usepackage{siunitx}                    % Consistent number / unit formatting
\sisetup{
  detect-all,
  separate-uncertainty = true,
  multi-part-units     = single,
}

% --- Algorithms -------------------------------------------------------------
\usepackage[ruled,vlined,linesnumbered]{algorithm2e}

% --- Code listings (optional) -----------------------------------------------
\usepackage{listings}
\lstset{
  basicstyle   = \ttfamily\small,
  keywordstyle = \bfseries\color{blue!70!black},
  commentstyle = \itshape\color{green!50!black},
  breaklines   = true,
  frame        = single,
  numbers      = left,
  numberstyle  = \tiny\color{gray},
}

% --- Bibliography -----------------------------------------------------------
\usepackage{natbib}
\bibliographystyle{plainnat}

% --- Cross-referencing (load after most other packages) ---------------------
\usepackage[
  colorlinks  = true,
  linkcolor   = blue!60!black,
  citecolor   = green!40!black,
  urlcolor    = magenta!70!black,
]{hyperref}
\usepackage[nameinlink,capitalise,noabbrev]{cleveref}

% --- Utilities --------------------------------------------------------------
\usepackage{enumitem}                   % Customisable lists
\usepackage{xspace}                     % Smart spacing after macros
\usepackage{lipsum}                     % Dummy text (remove in final version)

% --- Custom commands --------------------------------------------------------
% Shorthand for common math operators
\DeclareMathOperator*{\argmin}{arg\,min}
\DeclareMathOperator*{\argmax}{arg\,max}
\DeclareMathOperator{\diag}{diag}
\DeclareMathOperator{\tr}{tr}
\DeclareMathOperator{\rank}{rank}
\newcommand{\given}{\,\vert\,}          % Conditional bar: P(A \given B)
\newcommand{\R}{\mathbb{R}}
\newcommand{\N}{\mathbb{N}}
\newcommand{\E}{\mathbb{E}}
\newcommand{\Var}{\operatorname{Var}}
\newcommand{\Cov}{\operatorname{Cov}}

% Convenience
\newcommand{\eg}{e.g.\@\xspace}
\newcommand{\ie}{i.e.\@\xspace}
\newcommand{\cf}{cf.\@\xspace}
\newcommand{\etal}{et~al.\@\xspace}

% TODO / note macros (disable for camera-ready by redefining to {})
\newcommand{\todo}[1]{\textcolor{red}{\textbf{[TODO:} #1\textbf{]}}}
\newcommand{\note}[1]{\textcolor{blue}{\textbf{[NOTE:} #1\textbf{]}}}

% Theorem-like environments
\theoremstyle{plain}
\newtheorem{theorem}{Theorem}[section]
\newtheorem{lemma}[theorem]{Lemma}
\newtheorem{proposition}[theorem]{Proposition}
\newtheorem{corollary}[theorem]{Corollary}
\theoremstyle{definition}
\newtheorem{definition}[theorem]{Definition}
\theoremstyle{remark}
\newtheorem{remark}[theorem]{Remark}

% =============================================================================
% End of preamble
% =============================================================================

% \usepackage[top=2.5cm,bottom=2.5cm,left=2.5cm,right=2.5cm]{geometry}
% \usepackage{setspace}\onehalfspacing
% \title{Supplementary Materials}
% \author{}
% \date{}
% \begin{document}
% \maketitle
% ---

% =============================================================================
%  Reset counters for supplementary numbering (S1, S2, ...)
% =============================================================================
\setcounter{section}{0}
\setcounter{figure}{0}
\setcounter{table}{0}
\setcounter{equation}{0}

\renewcommand{\thesection}{S\arabic{section}}
\renewcommand{\thefigure}{S\arabic{figure}}
\renewcommand{\thetable}{S\arabic{table}}
\renewcommand{\theequation}{S\arabic{equation}}

% Update cleveref names
\crefname{section}{Supplementary Section}{Supplementary Sections}
\crefname{figure}{Supplementary Figure}{Supplementary Figures}
\crefname{table}{Supplementary Table}{Supplementary Tables}
\crefname{equation}{Supplementary Equation}{Supplementary Equations}

\clearpage
\begin{center}
  {\LARGE\bfseries Supplementary Materials}\\[12pt]
  {\large A Descriptive Title That Clearly Communicates
   the Main Finding of This Study}
\end{center}

\bigskip
\tableofcontents

% =============================================================================
\section{Supplementary Methods}
\label{sec:supp:methods}
% =============================================================================

% Extended methodological detail that does not fit in the main text.
% Include derivations, parameter choices, preprocessing steps, etc.

\subsection{Parameter Selection}
\label{sec:supp:methods:params}

% Describe hyperparameter tuning, cross-validation, grid search, etc.

\subsection{Detailed Derivations}
\label{sec:supp:methods:derivations}

% Full mathematical derivations for key results.

% =============================================================================
\section{Supplementary Figures}
\label{sec:supp:figures}
% =============================================================================

% \begin{figure}[htbp]
%   \centering
%   \includegraphics[width=0.85\textwidth]{figures/supp_placeholder.pdf}
%   \caption{%
%     \textbf{Supplementary Figure S1: Extended analysis.}
%     Detailed caption describing what the figure shows, how it was generated,
%     and how to interpret it.
%   }
%   \label{fig:supp:example}
% \end{figure}

% =============================================================================
\section{Supplementary Tables}
\label{sec:supp:tables}
% =============================================================================

% \begin{table}[htbp]
%   \centering
%   \caption{%
%     \textbf{Supplementary Table S1: Full benchmark results.}
%     Complete results for all methods and datasets.
%   }
%   \label{tab:supp:benchmark}
%   \begin{tabular}{@{} l *{4}{S[table-format=2.1]} @{}}
%     \toprule
%     {Method} & {Dataset A} & {Dataset B} & {Dataset C} & {Dataset D} \\
%     \midrule
%     Baseline   & 72.3 & 68.1 & 75.4 & 70.9 \\
%     Variant 1  & 78.5 & 74.2 & 80.1 & 76.3 \\
%     Variant 2  & 81.2 & 77.8 & 83.6 & 79.1 \\
%     Our method & 89.7 & 85.4 & 91.2 & 87.5 \\
%     \bottomrule
%   \end{tabular}
% \end{table}

% =============================================================================
\section{Supplementary Notes}
\label{sec:supp:notes}
% =============================================================================

% Additional discussion, edge cases, or extended analysis.

% --- Uncomment if compiling standalone --------------------------------------
% \end{document}
 at the end of main.tex,
%         or compile standalone (uncomment the preamble block below).
%
% =============================================================================

% --- Uncomment the following block to compile as a standalone document ------
% \documentclass[11pt,a4paper]{article}
% % =============================================================================
% preamble.tex — Shared preamble for research manuscripts
% =============================================================================
% Include via % =============================================================================
% preamble.tex — Shared preamble for research manuscripts
% =============================================================================
% Include via \input{preamble} BEFORE \begin{document}
% =============================================================================

% --- Encoding and fonts -----------------------------------------------------
\usepackage[utf8]{inputenc}
\usepackage[T1]{fontenc}
\usepackage{lmodern}                    % Latin Modern (scalable CM replacement)
\usepackage{microtype}                  % Subliminal refinements to typesetting

% --- Mathematics ------------------------------------------------------------
\usepackage{amsmath,amssymb,amsthm}
\usepackage{mathtools}                  % Extends amsmath (dcases, coloneqq, ...)
\usepackage{bm}                         % Bold math symbols via \bm{}

% --- Tables -----------------------------------------------------------------
\usepackage{booktabs}                   % \toprule, \midrule, \bottomrule
\usepackage{multirow}                   % Multi-row cells
\usepackage{array}                      % Extended column definitions

% --- Figures and graphics ---------------------------------------------------
\usepackage{graphicx}
\usepackage{xcolor}                     % Colour support (loaded before tikz)
\usepackage{subcaption}                 % Sub-figures with individual captions

% --- Units and numbers ------------------------------------------------------
\usepackage{siunitx}                    % Consistent number / unit formatting
\sisetup{
  detect-all,
  separate-uncertainty = true,
  multi-part-units     = single,
}

% --- Algorithms -------------------------------------------------------------
\usepackage[ruled,vlined,linesnumbered]{algorithm2e}

% --- Code listings (optional) -----------------------------------------------
\usepackage{listings}
\lstset{
  basicstyle   = \ttfamily\small,
  keywordstyle = \bfseries\color{blue!70!black},
  commentstyle = \itshape\color{green!50!black},
  breaklines   = true,
  frame        = single,
  numbers      = left,
  numberstyle  = \tiny\color{gray},
}

% --- Bibliography -----------------------------------------------------------
\usepackage{natbib}
\bibliographystyle{plainnat}

% --- Cross-referencing (load after most other packages) ---------------------
\usepackage[
  colorlinks  = true,
  linkcolor   = blue!60!black,
  citecolor   = green!40!black,
  urlcolor    = magenta!70!black,
]{hyperref}
\usepackage[nameinlink,capitalise,noabbrev]{cleveref}

% --- Utilities --------------------------------------------------------------
\usepackage{enumitem}                   % Customisable lists
\usepackage{xspace}                     % Smart spacing after macros
\usepackage{lipsum}                     % Dummy text (remove in final version)

% --- Custom commands --------------------------------------------------------
% Shorthand for common math operators
\DeclareMathOperator*{\argmin}{arg\,min}
\DeclareMathOperator*{\argmax}{arg\,max}
\DeclareMathOperator{\diag}{diag}
\DeclareMathOperator{\tr}{tr}
\DeclareMathOperator{\rank}{rank}
\newcommand{\given}{\,\vert\,}          % Conditional bar: P(A \given B)
\newcommand{\R}{\mathbb{R}}
\newcommand{\N}{\mathbb{N}}
\newcommand{\E}{\mathbb{E}}
\newcommand{\Var}{\operatorname{Var}}
\newcommand{\Cov}{\operatorname{Cov}}

% Convenience
\newcommand{\eg}{e.g.\@\xspace}
\newcommand{\ie}{i.e.\@\xspace}
\newcommand{\cf}{cf.\@\xspace}
\newcommand{\etal}{et~al.\@\xspace}

% TODO / note macros (disable for camera-ready by redefining to {})
\newcommand{\todo}[1]{\textcolor{red}{\textbf{[TODO:} #1\textbf{]}}}
\newcommand{\note}[1]{\textcolor{blue}{\textbf{[NOTE:} #1\textbf{]}}}

% Theorem-like environments
\theoremstyle{plain}
\newtheorem{theorem}{Theorem}[section]
\newtheorem{lemma}[theorem]{Lemma}
\newtheorem{proposition}[theorem]{Proposition}
\newtheorem{corollary}[theorem]{Corollary}
\theoremstyle{definition}
\newtheorem{definition}[theorem]{Definition}
\theoremstyle{remark}
\newtheorem{remark}[theorem]{Remark}

% =============================================================================
% End of preamble
% =============================================================================
 BEFORE \begin{document}
% =============================================================================

% --- Encoding and fonts -----------------------------------------------------
\usepackage[utf8]{inputenc}
\usepackage[T1]{fontenc}
\usepackage{lmodern}                    % Latin Modern (scalable CM replacement)
\usepackage{microtype}                  % Subliminal refinements to typesetting

% --- Mathematics ------------------------------------------------------------
\usepackage{amsmath,amssymb,amsthm}
\usepackage{mathtools}                  % Extends amsmath (dcases, coloneqq, ...)
\usepackage{bm}                         % Bold math symbols via \bm{}

% --- Tables -----------------------------------------------------------------
\usepackage{booktabs}                   % \toprule, \midrule, \bottomrule
\usepackage{multirow}                   % Multi-row cells
\usepackage{array}                      % Extended column definitions

% --- Figures and graphics ---------------------------------------------------
\usepackage{graphicx}
\usepackage{xcolor}                     % Colour support (loaded before tikz)
\usepackage{subcaption}                 % Sub-figures with individual captions

% --- Units and numbers ------------------------------------------------------
\usepackage{siunitx}                    % Consistent number / unit formatting
\sisetup{
  detect-all,
  separate-uncertainty = true,
  multi-part-units     = single,
}

% --- Algorithms -------------------------------------------------------------
\usepackage[ruled,vlined,linesnumbered]{algorithm2e}

% --- Code listings (optional) -----------------------------------------------
\usepackage{listings}
\lstset{
  basicstyle   = \ttfamily\small,
  keywordstyle = \bfseries\color{blue!70!black},
  commentstyle = \itshape\color{green!50!black},
  breaklines   = true,
  frame        = single,
  numbers      = left,
  numberstyle  = \tiny\color{gray},
}

% --- Bibliography -----------------------------------------------------------
\usepackage{natbib}
\bibliographystyle{plainnat}

% --- Cross-referencing (load after most other packages) ---------------------
\usepackage[
  colorlinks  = true,
  linkcolor   = blue!60!black,
  citecolor   = green!40!black,
  urlcolor    = magenta!70!black,
]{hyperref}
\usepackage[nameinlink,capitalise,noabbrev]{cleveref}

% --- Utilities --------------------------------------------------------------
\usepackage{enumitem}                   % Customisable lists
\usepackage{xspace}                     % Smart spacing after macros
\usepackage{lipsum}                     % Dummy text (remove in final version)

% --- Custom commands --------------------------------------------------------
% Shorthand for common math operators
\DeclareMathOperator*{\argmin}{arg\,min}
\DeclareMathOperator*{\argmax}{arg\,max}
\DeclareMathOperator{\diag}{diag}
\DeclareMathOperator{\tr}{tr}
\DeclareMathOperator{\rank}{rank}
\newcommand{\given}{\,\vert\,}          % Conditional bar: P(A \given B)
\newcommand{\R}{\mathbb{R}}
\newcommand{\N}{\mathbb{N}}
\newcommand{\E}{\mathbb{E}}
\newcommand{\Var}{\operatorname{Var}}
\newcommand{\Cov}{\operatorname{Cov}}

% Convenience
\newcommand{\eg}{e.g.\@\xspace}
\newcommand{\ie}{i.e.\@\xspace}
\newcommand{\cf}{cf.\@\xspace}
\newcommand{\etal}{et~al.\@\xspace}

% TODO / note macros (disable for camera-ready by redefining to {})
\newcommand{\todo}[1]{\textcolor{red}{\textbf{[TODO:} #1\textbf{]}}}
\newcommand{\note}[1]{\textcolor{blue}{\textbf{[NOTE:} #1\textbf{]}}}

% Theorem-like environments
\theoremstyle{plain}
\newtheorem{theorem}{Theorem}[section]
\newtheorem{lemma}[theorem]{Lemma}
\newtheorem{proposition}[theorem]{Proposition}
\newtheorem{corollary}[theorem]{Corollary}
\theoremstyle{definition}
\newtheorem{definition}[theorem]{Definition}
\theoremstyle{remark}
\newtheorem{remark}[theorem]{Remark}

% =============================================================================
% End of preamble
% =============================================================================

% \usepackage[top=2.5cm,bottom=2.5cm,left=2.5cm,right=2.5cm]{geometry}
% \usepackage{setspace}\onehalfspacing
% \title{Supplementary Materials}
% \author{}
% \date{}
% \begin{document}
% \maketitle
% ---

% =============================================================================
%  Reset counters for supplementary numbering (S1, S2, ...)
% =============================================================================
\setcounter{section}{0}
\setcounter{figure}{0}
\setcounter{table}{0}
\setcounter{equation}{0}

\renewcommand{\thesection}{S\arabic{section}}
\renewcommand{\thefigure}{S\arabic{figure}}
\renewcommand{\thetable}{S\arabic{table}}
\renewcommand{\theequation}{S\arabic{equation}}

% Update cleveref names
\crefname{section}{Supplementary Section}{Supplementary Sections}
\crefname{figure}{Supplementary Figure}{Supplementary Figures}
\crefname{table}{Supplementary Table}{Supplementary Tables}
\crefname{equation}{Supplementary Equation}{Supplementary Equations}

\clearpage
\begin{center}
  {\LARGE\bfseries Supplementary Materials}\\[12pt]
  {\large A Descriptive Title That Clearly Communicates
   the Main Finding of This Study}
\end{center}

\bigskip
\tableofcontents

% =============================================================================
\section{Supplementary Methods}
\label{sec:supp:methods}
% =============================================================================

% Extended methodological detail that does not fit in the main text.
% Include derivations, parameter choices, preprocessing steps, etc.

\subsection{Parameter Selection}
\label{sec:supp:methods:params}

% Describe hyperparameter tuning, cross-validation, grid search, etc.

\subsection{Detailed Derivations}
\label{sec:supp:methods:derivations}

% Full mathematical derivations for key results.

% =============================================================================
\section{Supplementary Figures}
\label{sec:supp:figures}
% =============================================================================

% \begin{figure}[htbp]
%   \centering
%   \includegraphics[width=0.85\textwidth]{figures/supp_placeholder.pdf}
%   \caption{%
%     \textbf{Supplementary Figure S1: Extended analysis.}
%     Detailed caption describing what the figure shows, how it was generated,
%     and how to interpret it.
%   }
%   \label{fig:supp:example}
% \end{figure}

% =============================================================================
\section{Supplementary Tables}
\label{sec:supp:tables}
% =============================================================================

% \begin{table}[htbp]
%   \centering
%   \caption{%
%     \textbf{Supplementary Table S1: Full benchmark results.}
%     Complete results for all methods and datasets.
%   }
%   \label{tab:supp:benchmark}
%   \begin{tabular}{@{} l *{4}{S[table-format=2.1]} @{}}
%     \toprule
%     {Method} & {Dataset A} & {Dataset B} & {Dataset C} & {Dataset D} \\
%     \midrule
%     Baseline   & 72.3 & 68.1 & 75.4 & 70.9 \\
%     Variant 1  & 78.5 & 74.2 & 80.1 & 76.3 \\
%     Variant 2  & 81.2 & 77.8 & 83.6 & 79.1 \\
%     Our method & 89.7 & 85.4 & 91.2 & 87.5 \\
%     \bottomrule
%   \end{tabular}
% \end{table}

% =============================================================================
\section{Supplementary Notes}
\label{sec:supp:notes}
% =============================================================================

% Additional discussion, edge cases, or extended analysis.

% --- Uncomment if compiling standalone --------------------------------------
% \end{document}


\end{document}
